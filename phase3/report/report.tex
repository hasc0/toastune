\documentclass{cse321}

\title{Term Project: toastune}
\author{Harper Scott}
\date{December 1, 2025}

\begin{document}

\maketitle

\section{Introduction}
The objective of \verb|toastune| is to make it possible to get a consistent piece of toast regardless
of what toaster is being used. The long-term ideal for this project would be a self contained hardware
unit that can be attached to any toaster, allowing any user to easily achieve their perfect toast profile.
In creating this prototype several challenges have appeared, most of which are software-related. However,
the prototype has been finalized with the exception of a consistent mounting system and a positional servo.

\section{Project Title and Problem Definition}
\begin{itemize}
    \item[a.]
    \textbf{Project Title:} toastune
    \item[b.]
    \textbf{Problem Definition:} An embedded system to stop any toaster once a chosen toast darkness
    has been reached, producing toast finished to one's personal preferences regardless of the toaster
    being used.
    \item[c.]
    \textbf{Functional Requirements:}
    \begin{list}{$\bullet$}{}
        \item
        The system will have three user selectable states (light, medium, and dark).
        \item
        Each state corresponds with a toast surface temperature, the lowest temperature being associated
        with light and the highest with dark.
        \item
        The system will include a display to show the selected state and a button to cycle through the
        three states.
        \item
        The system will have a button that will be pressed by the user when the toaster lever is pressed
        down to begin a toasting cycle.
        \item
        A thermocouple will be used to constantly measure the surface temperature of the toast.
        \item
        The system will employ a servomotor to actuate the button to end the toast cycle on the toaster
        once the selected temperature has been attained.
    \end{list}
    \item[d.]
    \textbf{Non-Functional Requirements:}
    \begin{list}{$\bullet$}{}
        \item 
        \textbf{Real-Time:}
        The system must be firm real-time, as missing the deadline results in overdone toast relative to
        the user's preference and as a result is discarded.
        \item 
        \textbf{Reliability:}
        Must be able to operate whenever the user decides to run the system to make toast and should not
        crash or lock up during use.
        \item 
        \textbf{Performance:}
        User inputs must be provided a visual response on a display within $500 \text{ ms}$, the
        thermocouple must provide a current value every $50 \text{ ms}$ at most, and the servomotor must
        actuate within $200 \text{ ms}$ of the selected temperature being reached.
    \end{list}
\end{itemize}

\section{Hardware Components}
The core of \verb|toastune| is an Arduino UNO Rev3. The final peripheral components are as follows:
\begin{itemize}
    \item USB-A to USB-B Cable
    \item ELEGOO Breadboard
    \item ELEGOO 0.96" 128x64 I$^2$C OLED Display (SSD1306)
    \item Feetech FS90R Continuous Rotation Servo
    \item Adafruit AD8495 K-Type Thermocouple Amplifier
    \item Adafruit Type-K Glass Braid Insulated Termocouple
    \item 18x Jumper Wire
    \item 2x Push Button
\end{itemize}
The purpose of the USB cable is to power the Arduino. All additional components recieve power from the 5V
pin on the Arduino itself. The OLED display serves to display the status of the system and the current
selected toast "darkness" setting alongside idle/run state and thermocouple temperature. The servo actuates
the button on the toaster that cancels the toasting process once the target temperature has been reached.
The thermocouple and amplifier are used to probe the surface of the bread in order to get the current
temperature reading. One of the buttons starts the system and the second button rotates through the toast
preference modes.

\section{Software Components}
The codebase is hosted on GitHub in order to maintain proper versioning to ensure that rollbacks are
possible. The URL for the repository is \url{https://github.com/hasc0/toastune}. The libraries used are
as follows:
\begin{itemize}
    \item Arduino
    \item U8x8
    \item Servo
\end{itemize}
The most obscure libary listed, U8x8, is a lightweight library used for producing output to the SSD1306
OLED display. This was chosen because the standard Adafruit library consumed a significant amount of
storage and memory and most of its functionality was not needed. U8x8 only allows for text output, which
is all that is needed for this project. The code is structured as minimally as possible within the core
\verb|loop()| function, as the FreeRTOS library consumed too much memory to be viable. The primary helper
functions, \verb|temperatureMonitor()| and \verb|displayOutput()|, serve to get the current thermocouple
temperature in Celsius and format all information for output to the display, respectively. Both helper
functions are called at least once per iteration of \verb|loop()|. The \verb|setup()| function contains
code to verify all output components are working. This is done by displaying a welcome message on the
display and briefly actuating the servo. Note that the Arduino CLI, Visual Studio Code, and the Arduino
Community Edition extension were used for development instead of the Arduino IDE.

\section{Circuit Design}
\begin{center}
    \includegraphics[width=0.6\linewidth]{figures/cd.png}
\end{center}

\section{Finite State Machine}
\begin{center}
    \includegraphics[width=\linewidth]{figures/fsm.png}
\end{center}

\section{Testing Scripts and Patterns}
The bulk of the testing is done manually using human input. The testing process is as follows:
\begin{itemize}
    \item
    On system reset, verify that the welcome message (project name, course name, and author name) all
    appear on screen.
    \item
    Verify that the display switches to real-time display of currently selected mode ("Light" following
    reset), current temperature in celsius, and the current system status ("Idle" following reset).
    \item
    Press the mode select button several time, ensuring that the select mode cycles from "Light" to
    "Medium" to "Dark" and back to "Light".
    \item
    Use a heat source to heat the thermocouple, confirm the temperature value approaches 750.
    \item
    Press the start button and confirm the status value changes from "Idle" to "Run".
    \item
    Heat the thermocouple and verify the servomotor actuates once the temperature value reaches 155, 170,
    and 185 for "Light", "Medium", and "Dark" modes, respectively.
\end{itemize}

\section{Observation and Notes}
First, the task structure previously utilized is highly volatile and experienced significant bugs. Due to
the size of some of the libraries used, stack overflows are frequent and the system often failed to
initialize. To rectify this, the code was restructed into a simple \verb|loop()|. The result has been a
more responsive system with significantly increased reliability. The second issue, which was unanticipated
due to poor planning, is that the use of a continuous servo makes it far more challenging to precisely
position the servo head. Instead of specifying a rotation angle, a formula must be used to convert the
speed at which the servo rotates to a rotation angle over a certain duration. This solution failed to
produce satisfactory results, so unfortunately the only acceptable solution is to replace the continuous
servo with a positional servo. The final issue pertains the the display library used. In initial testing,
the Adafruit SSD1306 library was used and produced reliable output. However, due to size constraints it was
replaced with U8x8 which is significantly lighter-weight.

\section{Conclusion}
There are some issues with the prototype that need to be solved, however the system in its current state
sufficiently functions as a well-rounded proof of concept. The prototype hardware is final with the exception
of the servo and the software, while more limited in scope and features than initially intended, works reliably
and responsively. The resulting toast is somewhat inconsistent due and relies heavily on thermocouple position,
however it typically produces an acceptable result and the system meets all required deadlines based on the
inputs.

\end{document}
