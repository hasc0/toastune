\documentclass{cse321}

\title{Term Project: toastune}
\author{Harper Scott}
\date{September 26, 2025}

\begin{document}

\maketitle

\section{Project Title and Problem Definition}
\begin{itemize}
    \item[a.]
    \textbf{Project Title:} toastune
    \item[b.]
    \textbf{Problem Definition:} An embedded system to stop any toaster once a chosen toast darkness
    has been reached, producing toast finished to one's personal preferences regardless of the toaster
    being used.
    \item[c.]
    \textbf{Functional Requirements:}
    \begin{list}{$\bullet$}{}
        \item
        The system will have three user selectable states (light, medium, and dark).
        \item
        Each state corresponds with a toast surface temperature, the lowest temperature being associated
        with light and the highest with dark.
        \item
        The system will include a display to show the selected state and a button to cycle through the
        three states.
        \item
        The system will include an additional display to show current surface temperature and duration
        of current toast cycle.
        \item
        The system will have a button that will be pressed when the toaster lever is pressed down to begin
        a toasting cycle.
        \item
        A thermocouple will be used to constantly measure the surface temperature of the toast.
        \item
        The system will employ a servomotor to actuate the button to end the toast cycle on the toaster
        once the selected temperature has been attained.
    \end{list}
    \item[d.]
    \textbf{Non-Functional Requirements:}
    \begin{list}{$\bullet$}{}
        \item 
        \textbf{Real-Time:}
        The system must be firm real-time, as missing the deadline results in overdone toast relative to
        the user's preference and as a result is discarded.
        \item 
        \textbf{Reliability:}
        Must be able to operate whenever the user decides to run the system to make toast and should not
        crash or lock up during use.
        \item 
        \textbf{Performance:}
        User inputs must be provided a visual response on a display within $500 \text{ ms}$, the
        thermocouple must provide a current value every $50 \text{ ms}$ at most, and the servomotor must
        actuate within $200 \text{ ms}$ of the selected temperature being reached.
    \end{list}
\end{itemize}

\section{UML Use Case Diagram}
\begin{center}
    \includegraphics[width=\linewidth]{figures/uc.png}
\end{center}

\section{CRC Cards}
\begin{center}
    \includegraphics[width=0.6\linewidth]{figures/crc.png}
\end{center}

\section{Finite State Machine Design}
% \begin{center}
%     \includegraphics[width=\linewidth]{figures/fsm.png}
% \end{center}

\section{Components List}
\begin{itemize}
    \item
    \textbf{Hardware:}
    \begin{list}{$\bullet$}{}
        \item
        1x Arduino UNO Rev3: On Hand
        \item
        2x ELEGOO 0.96 Inch OLED Display Module: On Hand
        \item
        1x Adafruit AD8495: On Hand
        \item
        1x Type-K Thermocouple: On Hand
        \item
        1x FS90R Servomotor: On Hand
        \item
        1x Power Supply: On Hand
        \item
        2x Button: On Hand
    \end{list}
    \item
    \textbf{Software:}
    \begin{list}{$\bullet$}{}
        \item
        FreeRTOS Library
        \item
        Display Driver Library
        \item
        Thermocouple Library
        \item
        Visual Studio Code/Neovim
        \item
        Wokwi Simulator (Testing)
    \end{list}
\end{itemize}

\end{document}
