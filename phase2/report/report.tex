\documentclass{cse321}

\title{Term Project: toastune}
\author{Harper Scott}
\date{November 12, 2025}

\begin{document}

\maketitle

\section{Introduction}
The objective of \verb|toasttune| is to make it possible to get a consistent piece of toast regardless
of what toaster is being used. The long-term ideal for this project would be a self contained hardware
unit that can be attached to any toaster, allowing any user to easily achieve their perfect toast profile.
In creating this prototype several challenges have appeared, most of which are software-related. However,
the hardware has been finalized with the exception of a consistent mounting system.

\section{Hardware Components}
The core of \verb|toasttune| is an Arduino UNO R3. The final peripheral components are as follows:
\begin{itemize}
    \item Arduino UNO R3
    \item USB-A to USB-B Cable
    \item ELEGOO Breadboard
    \item ELEGOO 0.96" 128x64 I$^2$C OLED Display (SSD1306)
    \item Feetech FS90R Continuous Rotation Servo
    \item Adafruit AD8495 K-Type Thermocouple Amplifier
    \item Adafruit Type-K Glass Braid Insulated Termocouple
    \item 18x Jumper Wire
    \item 2x Button
\end{itemize}
The purpose of the USB cable is to power the Arduino. All additional components recieve power from the 5V
pin on the Arduino itself. The OLED display serves to display the status of the system and the current
selected toast "darkness" setting alongside elapsed time and thermocouple when the system is running
temperature. The servo actuates the button on the toaster that cancels the toasting process once the
target temperature has been reached. The thermocouple and amplifier are used to probe the surface of the
bread in order to get the current temperature reading. One of the buttons starts and stops the system and
the second button rotates through the toast preference modes.

\pagebreak

\section{Software Components}
The codebase is hosted on GitHub in order to maintain proper versioning to ensure that rollbacks are
possible in situations where major breakages occur. The libraries used are as follows:
\begin{itemize}
    \item FreeRTOS (\verb|queue.h|/\verb|task.h|)
    \item U8X8
    \item Wire
    \item Servo
\end{itemize}
The most obscure libary listed, U8X8, is a lightweight library used for producing output to the SSD1306
OLED display. This was chosen because the standard Adafruit library consumed a significant amount of
storage and memory and most of its functionality was not needed. U8X8 only allows for text output, which
is all that is needed for this project. The code is currently organized into three major task components:
\begin{itemize}
    \item \verb|inputHandler|
    \item \verb|displayManager|
    \item \verb|toastService|
\end{itemize}
The \verb|inputHandler| task processes all button inputs and pushes relevant events to the event queue.
The \verb|displayManager| task handles all output to the display, including the formatting of strings
and reading the thermocouple input in order to display an accurate, up-to-date reading. Finally, the
\verb|toastService| task handles the actual toasting process and actuation of the servo, along with
management of events and state as processed from the inputs. The \verb|setup()| function contains code
to verify all output components are working. This is done by displaying a welcome message on the display
and briefly actuating the servo.

\section{Testing Scripts and Patterns}
The bulk of the testing is done manually using human input. The testing process is as follows:
\begin{itemize}
    \item
    On system reset, verify that the welcome message (project name, course name, and author name) all
    appear on screen.
    \item
    Verify that the display switches to real-time display of currently selected mode ("Light" following
    reset), current temperature in celsius, and the elapsed time ("Idle" following reset).
    \item
    Press the mode select button several time, ensuring that the select mode cycles from "Light" to
    "Medium" to "Dark" and back to "Light".
    \item
    Use a heat source to heat the thermocouple, confirm the temperature value approaches 750.
    \item
    Press the start button and confirm the time value changes from "Idle" to an elapsed seconds value.
    \item
    Heat the thermocouple and verify the display ouputs confirmation of toast process conclusion once
    the temperature value reaches 200, 220, and 240 for "Light", "Medium", and "Dark" modes, respectively.
    \item
    Confirm that the servo actuated with every completed toasting process.
    \item
    Start the toasting cycle then cancel it and confirm the servo actuates.
\end{itemize}

\section{Images}
\begin{center}
    \includegraphics[width=\linewidth]{figures/wide.png}
\end{center}

\begin{center}
    \includegraphics[width=\linewidth]{figures/close.png}
\end{center}

\section{Observation and Notes}
The main observation is that the task structure currently utilized is highly volatile and experiences
significant bugs. Due to the size of some of the libraries used, stack overflows are frequent and the
system often fails to initialize. To rectify this, the idea is to refactor the structure of the code in
a way that eliminates tasks. The \verb|toastService| will serve as the core of the loop and will call
other functions in order to function correctly. The anticipated result of this is a slightly less
responsive system with significantly increased reliability. The second issue, which was unanticipated
due to poor planning, is that the use of a continuous servo makes it far more challenging to precisely
position the servo head. Instead of specifying a rotation angle, a formula must be used to convert the
speed at which the servo rotates to a rotation angle over a certain duration. This solution is less
repeatable but will work within a reasonable margin of error. The final issue pertains the the display
library used. In initial testing, the Adafruit SSD1306 library was used and produced reliable output.
However, due to size constraints it was replaced with U8X8 which is significantly lighter-weight. This
library seems to interact poorly with the library for servo handling in some cases and occasionally
breaks servo output. It is still unclear if this issue is completely related to the library change, but
when this potential cause was isolated it seemed to be the root cause.

\section{Conclusion}
There are a few issues needing resolution in order to make the final product viable, but these are
relatively minor software changes. The hardware has been finalized and the system works in several cases,
so the remainder of the work is effectively fixing bugs and determining the final temperature bounds for
each preference mode.

\end{document}
